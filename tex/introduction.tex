%!TEX root = ../main.tex

% Slide: Introduction to Machine Learning
\section{Introduction to Machine Learning}
\begin{frame}{What is Machine Learning?}
    \begin{itemize}
        \item Machine learning is the study of algorithms that improve automatically through experience.
        \item Key applications include:
        \begin{itemize}
            \item Image recognition
            \item Natural language processing
            \item Predictive analytics
        \end{itemize}
    \end{itemize}
\end{frame}

% Slide: Introduction to Machine Learning
\begin{frame}{Supervised learning}
    \begin{tikzpicture}[
        node distance=1cm and 1.5cm,
        box/.style={rounded corners, draw, fill=blue!20, minimum width=2.5cm, minimum height=1cm, align=center},
        main/.style={rounded corners, draw, fill=red!60, text=white, font=\bfseries, minimum width=3cm, minimum height=1cm, align=center},
        arrow/.style={thick, -{Latex[length=2mm, width=2mm]}, blue!70}
    ]
    
    % Nodes
    \node (data) [box] {Training data};
    \node (labels) [box, left=of data] {Labels};
    \node (algorithm) [main, below=of data] {Machine learning algorithm};
    \node (model) [box, below=of algorithm] {Predictive model};
    \node (newdata) [box, left=of model] {New data};
    \node (prediction) [box, right=of model] {Predicted labels};
    
    % Arrows
    \draw [arrow] (labels) -- (algorithm);
    \draw [arrow] (data) -- (algorithm);
    \draw [arrow] (algorithm) -- (model);
    \draw [arrow] (newdata) -- (model);
    \draw [arrow] (model) -- (prediction);
    
    \end{tikzpicture}
\end{frame}

% Slide 4: Types of Machine Learning
\section{Types of Machine Learning}
\begin{frame}{Types of Machine Learning}
    \begin{itemize}
        \item \textbf{Supervised Learning}
        \begin{itemize}
            \item Uses labeled data
            \item Examples: Classification, Regression
        \end{itemize}
        \item \textbf{Unsupervised Learning}
        \begin{itemize}
            \item Uses unlabeled data
            \item Examples: Clustering, Dimensionality Reduction
        \end{itemize}
        \item \textbf{Reinforcement Learning}
        \begin{itemize}
            \item Learning through rewards and penalties
        \end{itemize}
    \end{itemize}
\end{frame}

% Slide 5: Key Concepts
\section{Key Concepts}
\begin{frame}{Key Concepts in Machine Learning}
    \begin{itemize}
        \item \textbf{Training Data}: Dataset used to train the model.
        \item \textbf{Test Data}: Dataset used to evaluate the model.
        \item \textbf{Overfitting}: When a model learns noise instead of the signal.
        \item \textbf{Regularization}: Techniques to prevent overfitting.
    \end{itemize}
\end{frame}

% Slide 6: Supervised Learning Example
\section{Supervised Learning Example}
\begin{frame}{Supervised Learning: Linear Regression}
    \begin{block}{Linear Regression}
        Linear regression aims to model the relationship between a scalar response and one or more explanatory variables.
        \[
        y = \beta_0 + \beta_1 x + \epsilon
        \]
    \end{block}
    \begin{itemize}
        \item \textbf{Goal}: Minimize the error between predicted and actual values.
        \item \textbf{Cost Function}: Mean Squared Error (MSE)
    \end{itemize}
\end{frame}

% Slide 7: Visualization
\section{Visualization}
% \begin{frame}{Visualizing Data}
%     \begin{figure}
%         \centering
%         \includegraphics[width=0.7\textwidth]{example_plot.png}
%         \caption{Sample visualization of data points with regression line}
%     \end{figure}
% \end{frame}

% Slide 8: Questions
\section{Questions}
\begin{frame}{Questions?}
    \begin{center}
        \Large Any Questions?
    \end{center}
\end{frame}